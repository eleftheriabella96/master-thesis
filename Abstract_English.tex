\en
\chapter*{Abstract}
The human heart, although it is the size of a human fist, manages to pump blood to the entire human body, continuously. Thanks to its contractions, it pushes the blood to the different organs while the 'unfiltered' blood returns to it so it can be cleaned and repeat the same cycle. This cycle is what we call a heart beat, the total of  which define the heart rhythm.  These contractions are possible because of low current signals that force the heart muscle to move.
\par
The alteration of heart beats can be depicted in a electrocardiogram, a medical test that aids the prevention of heart conditions and evaluation of the heart's physical state. The variability between these pulses is called heart rate variability and it defines an important index related to the heart's physical condition. What is worth mentioning that the recording performed during the electrocardiogram is often faulty, due to noise introduced to the signal because of outer and inner factors. 
\par
In cases that the pumping of blood deviates from normal, there is loss of normal heart beats and the function of the heart becomes defective. These events are called arrhythmias, and they pose no threat if not repeated. When they do, though, the patients needs to undergo tests in order to be prevented. Hearth arrhythmias are categorized in atrial, when they are generated in the atrias of the heart and ventricular when they occur in the ventricles. Moreover, they are also dividen into tachyarrhythmias, which are related to a higher heart rhythm and bradyarrhythmias, which are characterized by a slower heart rhtyhm.
\par
Due to the complex nature of arrhythmias and the factors that are involved in their morphology, it is imperative that tools for their evaluation and prevention are developed. For this goal to be accomplished, there are medical tests and algorithms that have been factored so they can filter out the noise from the heart signal that might alter it. They are also capable of applying metrics and methods to it so they can locate areas of abnormal heart activity.
\par
The following thesis aims to an algorithm development that is based on the use of widely accepted entropy and signal complexity estimation methods. Also, the studying of actual heart signals taken from actual people of different age, sex who might suffer from various heart conditions, gives a more comprehensive view of the algorithm's performance; the algorithm studies the each metric and method in small time-varying 'windows', which are signal portions. This algorithm aims to identify arrhythmias as accurately as possible, as its basis consists of real heart signal recordings of different heart diseases, which is the core of this study.