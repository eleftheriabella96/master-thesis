\gr
\chapter{Συμπέρασμα-Συζήτητση}
Έχοντας αναλύσει την ανθρώπινη καρδιά, τη λειτουργία της, τις παθήσεις της και τους τρόπους με τους οποίους αξιολογείται η λειτουργία της, προσπαθήσαμε να αναπτύξουμε έναν αλγόριθμο που εντοπίζει τις περιοχές του καρδιακού σήματος που υποδεικνύουν την ύπαρξη πιθανού κινδύνου. Όσον αφορά τις βιβλιοθήκες, τα σήματα που διαθέτουν εναλλαγές ανάμεσα σε φυσιολογική καρδιακή δραστηριότητα και φαινόμενα καρδιακών παθήσεων είναι αυτά που ευνοούν τον αλγόριθμο και αποδίδουν τα πιο έγκυρα αποτελέσματα. Αντίθετα, στις βιβλιοθήκες που τα δείγματα αποκλίνουν μόνιμα από φυσιολογική καρδιακή λειτουργία, τα αποτελέσματα δεν είναι πάντα τα επιθυμητά, δυσκολεύοντας τη διαδικασία εκτίμησης και εντοπισμού επικίνδυνων περιοχών. Από τη μία, τα φυσιολογικά σήματα δε διαθέτουν περιοχές ανησυχίας, επομένως όποια κι αν είναι η εκτίμηση του αλγορίθμου δεν είναι απαραίτητη η ύπαρξη επικίνδυνων περιοχών. Από την άλλη, όταν τα σήματα χαρακτηρίζονται από διαρκείς και μόνιμες αρρυθμίες, ακόμα κι αν ο αλγόριθμος εντοπίσει τις πιο υψηλές, η εκτίμηση δεν έχει την ίδια βαρύτητα. Το φιλτράρισμα του θορύβου, όμως, και η τεχνητή εισαγωγή  έκτοπων συστολών στις "φυσιολογικές" βιβλιοθήκες συμβάλλουν στη διεξαγωγή πιο ευνοϊκών και έγκυρων εκτιμήσεων.  Με εξαίρεση τις δύο συλλογές που απορρίφθηκαν λόγω ανεπάρκειας σχετικά με τη διάρκεια της καταγραφής, ο αλγόριθμος κατάφερε στην πλειοψηφία των περιπτώσεων να εστιάσει σε περιοχές καρδιακής δραστηριότητας που ενδέχεται να αποτελέσουν λόγο ανησυχίας και περαιτέρω διερεύνησης από τους ειδικούς.
\par
Οι μετρικές και οι μέθοδοι που χρησιμοποιήθηκαν αποτελούν τον πυρήνα της λειτουργίας του αλγορίθμου στην πλειοψηφία των υπό εξέταση σημάτων. Το σύνολό τους απαρτίζεται από τις πλέον ευρέως εφαρμοζόμενες μετρικές, επομένως δεν τίθεται αμφιβολία ως προς την εγκυρότητά τους ή την ορθότητα λειτουργιας τους. Παρόλα αυτά, δεν καταφέρνουν όλες να επιτύχουν το ζητούμενο στόχο, λόγω της κατασκευής τους, της μορφολογίας των σημάτων ακόμα και του μήκους των καταγραφών. Πιο συγκεκριμένα, οι τρεις μέθοδοι εκτίμησης εντροπίας του σήματος φαίνεται να μην αποδίδουν τόσο ικανοποιητικά. Αυτό συμβαίνει διότι για να μπορέσουν να εκτιμήσουν με μεγαλύτερη ακρίβεια τα αποτελέσματα και να καταλήξουν στο επιθυμητό αποτέλεσμα, απαιτούν σήματα με αρκετά μεγαλύτερη διάρκεια. Υπάρχουν, από την άλλη, μετρικές οι οποίες εκπλήρωσαν το στόχο τους, εδραιώνοντας τη θέση τους στο χώρο επεξεργασίας καρδιακών σημάτων. Το σημαντικό είναι πως οι συγκεκριμένες μετρικές (\en RMSSD, STD, Haar wavelets \gr)  έδειξαν την ίδια ικανότητα στο σύνολο των βιβλιοθηκών, χωρίς να επηρεάζονται σε μεγάλο βαθμό από μορφολογία, διάρκεια και αλλοίωση του σήματως λόγω κάποιας καρδιακής πάθησης. Για αυτό ακριβώς το λόγο συμπεραίνεται πως αποτελούν τους καταληλλότερους δείκτες όσον αφορά τη μελέτη καρδιακών σημάτων σχετικά με εντοπισμό αρρυθμιών, τουλάχιστον στα πλαίσια της συγκεκριμένης διπλωματικής εργασίας. 
\par
Τέλος, τα μεγέθη των παραθύρων στα οποία εκτιμώνται οι μετρικές φαίνεται να προσαρμόζονται σε κάθε περίπτωση. Υπήρξαν βιβλιοθήκες και κατηγορίες πειραμάτων που απέδωσαν με μικρά μήκη παραθύρων και περιπτώσεις στις οποίες το μέγιστο μήκος βοήθησε περισσότερο από ό,τι τα συντομότερης διάρκειας παράθυρα. Είναι ευκόλως εννοούμενο πως καταγραφές συντομότερης διάρκειας ευνοούνται από μικρότερα μήκη παραθύρων, μιας και τα μεγαλύτερα παράθυρα θα κάλυπταν αρκετό μήκος του σήματος, με αποτέλεσμα οι εκτιμήσεις να μην έχουν ιδιαίτερο νόημα. Όταν, βέβαια, η διάρκεια καταγραφής αυξάνετα σε υπερβολικό βαθμό, ακόμα και το μέγιστο παράθυρο δεν είναι αρκετό για μία ακριβή εκτιμηση. 
\par
Μία ακόμη περίπτωση στην οποία επιβεβαιώθηκε η ικανότητα -ή η έλλειψη αυτής- εντοπισμού των ζητούμενων περιοχών από τον αλγόριθμο αποτέλεσαν οι βιβλιοθήκες φυσιολογικών καταγραφών. Η συμβολή τους ήταν διττή, διότι από τη μία κάλυψαν την ανάγκη για περισσότερες κατηγορίες πειραμάτων (εκτιμήθηκαν σε μεγαλύτερη λεπτομέρεια περιπτώσεις έκτοπων συσπάσεων) και από την άλλη επιβαβαίωσαν τη λειτουργία του αλγορίθμου σε πιο "καθαρά" δείγματα, όπου η αναλλαγή ανάμεσα σε φυσιολογική -και μη- καρδιακή λειτουργία ήταν ξεκάθαρη. Παρόλα αυτά, τα αποτελέσματα αυτών των βιβλιοθηκών κάνουν το αποτέλεσμα πιο εύκολα διακριτό και στο μάτι, καθώς είναι πολύ γρηγορότερο να εντοπιστούν οι περιοχές που αποκλίνουν από το φυσιολογικό και να φανεί η απόδοση της κάθε μετρικής σε κάθε ένα από τα δείγματα.
\par
Ο στόχος της διπλωματικής αυτής εργασίας φαίνεται να έχει επιτευχθεί, τουλάχιστον ως ένα βαθμό. Αν και τα ποσοστά επιτυχίας δεν είναι απόλυτα, καταφέρνουν να εντοπίσουν περιοχές που αποκλίνουν από το μέσο όρο και τη μέση φυσιολογική καρδιακή λειτουργία. Προφανώς η απόδοση και επιτυχία του αλγορίθμου εξαρτάται σε μεγάλο βαθμό από τη μορφολογία των σημάτων και το ποσοστό θορύβου που περιέχεται σε αυτά, καθώς και η δυνατότητα απομάκρυνσής του θορύβου από τις καταγραφές. Υπήρξαν βιβλιοθήκες που απέτυχαν και άλλες που ήταν επιτυχείς και αυτό οφείλεται σε μεγάλο βαθμό στη μορφολογία των σημάτων, όπως έχει προαναφερθεί. Ο αλγόριθμος αυτός μπορεί να αποτελέσει μια βάση για μετέπειτα εξέλιξη, να γίνει πιο ακριβής και εύστοχος στις προβλέψεις του και να μειωθεί το περιθώριο λάθους που παρατηρείται, με στόχο ιδανικά το περιθώριο λάθους αυτό να εκμηδενιστεί. 