\gr
\chapter*{Εκτεταμένη Περίληψη}
Η ανθρώπινη καρδιά, αν και έχει μέγεθος ανθρώπινης γροθιάς, καταφέρνει να αντλεί αίμα σε ολόκληρο το σώμα χωρίς κανένα διάλειμμα. Μέσω των συσπάσεών της, η καρδιά ωθεί το αίμα προς τα διαφορετικά όργανα, ενώ το "αφιλτράριστο" αίμα επιστρέφει σε αυτήν ώστε να καθαριστεί και να επαναλάβει τον κύκλο του. Ένας τέτοιος κύκλος αποτελεί τον καρδιακό παλμό, με το σύνολο των τελευταίων να ορίζουν τον καρδιακό ρυθμό. Οι καρδιακές συσπάσεις γίνονται δυνατές μέσω ηλεκτρικών ώσεων χαμηλής έντασης οι οποίες εξαναγκάζουν τον μυ της καρδιάς σε κίνηση.
\par
Η μεταβολή των καρδιακών παλμών απεικονίζεται στο ηλεκτροκαρδιογράφημα, μια ιατρική εξέταση που συμβάλλει στην πρόληψη καρδιακών παθήσεων και στην αξιολόγηση της υγείας του μυοκαρδίου. Η διακύμανση των παλμών ονομάζεται μεταβλητότητα του καρδιακού παλμού και αποτελεί σημαντικό δείκτη για τη φυσική κατάσταση της καρδιάς. Μετράται με πολλές διαφορετικές μεθόδους, αρκετές εκ των οποίων στοχεύουν στην εκτίμηση της πολυπλοκότητας του καρδιακού παλμού, αποτελώντας, έτσι, ένα σημαντικό εργαλείο στην αξιολόγηση της καρδιακής λειτουργίας. Είναι ωστόσο σημαντικό να αναφερθεί πως η καταγραφή που πραγματοποιείται μέσω του ηλεκτροκαρδιογραφήματος είναι συχνά ατελής, καθώς εξωγενείς και ενδογενείς παράγοντες εισάγουν θόρυβο και αλλοιώνουν το τελικό αποπτέλεσμα.
\par
Στις περιπτώσεις στις οποίες η άντληση του αίματος παρουσιάζει αποκλίσεις, παρουσιάζεται απώλεια φυσιολογικού παλμού και ελαττωματική λειτουργία της καρδιάς. Οι δυσλειτουργίες αυτές ονομάζονται αρρυθμίες και όταν εμφανίζονται μεμονωμένα δεν αποτελούν κίνδυνο. Ωστόσο, η επανειλημμένη εμφάνισή τους χρίζει εξέτασης, προκειμένου να προληφθούν και να αποφευχθούν σοβαρές επιπλοκές. Οι καρδιακές αρρυθμίες ανάλογα με το τμήμα της καρδιάς από το οποίο ξεκινούν χωρίζονται σε δύο γενικές κατηγορίες: τις κολπικές, όταν ξεκινούν από τους κόλπους της καρδιάς και τις κοιλιακές, όταν ξεκινούν από τις κοιλίες. Επιπλέον, οι αρρυθμίες χωρίζονται σε ταχυκαρδίες, οι οποίες χαρακτηρίζονται από αύξηση του καρδιακού ρυθμού και βραδυκαρδίες, οι οποίες συνδέονται με μείωσή του.
\par
Λόγω της σύνθετης φύσης των αρρυθμιών και των διαφόρων παραγόντων που επηρεάζουν τη μορφολογία τους, κρίνεται επιτακτική η ανάπτυξη εργαλείων για αξιολόγηση και πρόληψη τους. Για την επίτευξη του παραπάνω στόχου, έχουν αναπτυχθεί εξετάσεις και αλγόριθμοι που «φιλτράρουν» το καρδιακό σήμα, απομακρύνοντας πηγές θορύβου που μπορεί να το αλλοιώνουν, εφαρμόζουν μετρικές και μεθόδους ανθεκτικές σε αυτόν και καταφέρνουν να εντοπίσουν σημεία αποκλίνουσας καρδιακής δραστηριότητας. 
\par
Η ακόλουθη εργασία έχει ως στόχο την σύνθεση ενός αλγορίθμου, ο οποίος βασίζεται στη χρήση ευρέως αποδεκτών μεθόδων εκτίμησης εντροπίας και πολυπλοκότητας του σήματος. Η χρήση πραγματικών καρδιακών σημάτων, από άτομα διαφορετικών ηλικιών, φύλων, με διαφορετικές καρδιακές παθήσεις που μελετώνται δίνει μια πιο ολοκληρωμένη άποψη για την απόδοση του αλγορίθμου, ο οποίος μελετά την πρόβλεψη κάθε μετρικής και μεθόδου σε μικρά χρονικά μεταβαλλόμενα «παράθυρα», τμήματα δηλαδή του σήματος. Η ανάπτυξη του αλγορίθμου αυτού αποσκοπεί στον εντοπισμό αρρυθμιών με όσο το δυνατόν υψηλότερη ακρίβεια, καθώς η βάση του αποτελείται από πραγματικές καταγραφές καρδιακών σημάτων διαφορετικών καρδιακών παθήσεων και αποτελεί τον πυρήνα μελέτης αυτής της διπλωματικής εργασίας.
